% Options for packages loaded elsewhere
\PassOptionsToPackage{unicode}{hyperref}
\PassOptionsToPackage{hyphens}{url}
%
\documentclass[
  ignorenonframetext,
  serif,
  professionalfont,
  usenames,
  dvipsnames,
  aspectratio = 169]{beamer}
\usepackage{pgfpages}
\setbeamertemplate{caption}[numbered]
\setbeamertemplate{caption label separator}{: }
\setbeamercolor{caption name}{fg=normal text.fg}
\beamertemplatenavigationsymbolsempty
% Prevent slide breaks in the middle of a paragraph
\widowpenalties 1 10000
\raggedbottom
\setbeamertemplate{part page}{
  \centering
  \begin{beamercolorbox}[sep=16pt,center]{part title}
    \usebeamerfont{part title}\insertpart\par
  \end{beamercolorbox}
}
\setbeamertemplate{section page}{
  \centering
  \begin{beamercolorbox}[sep=12pt,center]{part title}
    \usebeamerfont{section title}\insertsection\par
  \end{beamercolorbox}
}
\setbeamertemplate{subsection page}{
  \centering
  \begin{beamercolorbox}[sep=8pt,center]{part title}
    \usebeamerfont{subsection title}\insertsubsection\par
  \end{beamercolorbox}
}
\AtBeginPart{
  \frame{\partpage}
}
\AtBeginSection{
  \ifbibliography
  \else
    \frame{\sectionpage}
  \fi
}
\AtBeginSubsection{
  \frame{\subsectionpage}
}
\usepackage{amsmath,amssymb}
\usepackage{lmodern}
\usepackage{iftex}
\ifPDFTeX
  \usepackage[T1]{fontenc}
  \usepackage[utf8]{inputenc}
  \usepackage{textcomp} % provide euro and other symbols
\else % if luatex or xetex
  \usepackage{unicode-math}
  \defaultfontfeatures{Scale=MatchLowercase}
  \defaultfontfeatures[\rmfamily]{Ligatures=TeX,Scale=1}
\fi
% Use upquote if available, for straight quotes in verbatim environments
\IfFileExists{upquote.sty}{\usepackage{upquote}}{}
\IfFileExists{microtype.sty}{% use microtype if available
  \usepackage[]{microtype}
  \UseMicrotypeSet[protrusion]{basicmath} % disable protrusion for tt fonts
}{}
\makeatletter
\@ifundefined{KOMAClassName}{% if non-KOMA class
  \IfFileExists{parskip.sty}{%
    \usepackage{parskip}
  }{% else
    \setlength{\parindent}{0pt}
    \setlength{\parskip}{6pt plus 2pt minus 1pt}}
}{% if KOMA class
  \KOMAoptions{parskip=half}}
\makeatother
\usepackage{xcolor}
\newif\ifbibliography
\usepackage{graphicx}
\makeatletter
\def\maxwidth{\ifdim\Gin@nat@width>\linewidth\linewidth\else\Gin@nat@width\fi}
\def\maxheight{\ifdim\Gin@nat@height>\textheight\textheight\else\Gin@nat@height\fi}
\makeatother
% Scale images if necessary, so that they will not overflow the page
% margins by default, and it is still possible to overwrite the defaults
% using explicit options in \includegraphics[width, height, ...]{}
\setkeys{Gin}{width=\maxwidth,height=\maxheight,keepaspectratio}
% Set default figure placement to htbp
\makeatletter
\def\fps@figure{htbp}
\makeatother
\setlength{\emergencystretch}{3em} % prevent overfull lines
\providecommand{\tightlist}{%
  \setlength{\itemsep}{0pt}\setlength{\parskip}{0pt}}
\setcounter{secnumdepth}{-\maxdimen} % remove section numbering
% Definição do esquema de cores:
% 1. UFPR - Azul com cinza.
% 2. DEST - Roxo com cinza.
% 3. LEG - Laranjado com cinza.
\def\mycolorscheme{1}

% Caminho para a imagem de fundo com aspecto 16x9.
% \def\pathtobg{config/ufpr-fachada-baixo-1.jpg}
% \def\pathtobg{config/ufpr-fundo.jpg}
% \def\pathtobg{config/ufpr-fundo.jpg}
\def\pathtobg{./config/ufpr-fundo-16x9.jpg}

% \providecommand{\tightlist}{%
%   \setlength{\itemsep}{0pt}\setlength{\parskip}{0pt}}
% ATTENTION: Redefine o comando acima que é definido pelo template.
% \renewcommand{\tightlist}{}
\renewcommand{\tightlist}{%
  \setlength{\itemsep}{0\baselineskip}
  \setlength{\parskip}{0.25\baselineskip}
}

% Logo na capa.
\titlegraphic{
  %\vspace{-1em}
  %\includegraphics[height=1.2cm]{config/dest-texto-2.png}\hspace{1em}
  %\includegraphics[height=1.8cm]{config/dsbd-logo-2x2.png}\hspace{1em}
  \includegraphics[height=1.8cm]{config/ufpr-transparent-600px.png}
}
%-----------------------------------------------------------------------

% Palladio.
% \usepackage[sc]{mathpazo}
% \linespread{1.05}         % Palladio needs more leading (space between lines)
% \usepackage[T1]{fontenc}

% Kurier.
% \usepackage[light, condensed, math]{kurier}
% \usepackage[T1]{fontenc}

% Iwona.
% \usepackage[math, light, condensed]{iwona}

% \usepackage{cmbright}
% \usepackage[charter]{mathdesign}
% \usepackage{palatino}

% Roboto (with Iwona for maths).
% \usepackage[math]{iwona}
% \usepackage[sfdefault, light, condensed]{roboto}

% Source Sans Pro (with Iwona for maths).
% \usepackage[math]{iwona}
% \usepackage[default, light]{sourcesanspro}

% Lato (with Iwona for maths).
% \usepackage[math]{iwona}
% \usepackage[default]{lato}

% Fira Sans (with Iwona for maths).
\usepackage[math, light]{iwona}
\usepackage[sfdefault,light]{FiraSans} %% option 'sfdefault' activates Fira Sans as the default text font
\usepackage[T1]{fontenc}
\renewcommand*\oldstylenums[1]{{\firaoldstyle #1}}

% Font for code. ----------------------------
% \usepackage[scaled=.75]{beramono}
\usepackage{inconsolata}

% ATTENTION: needs complile with xelatex: `$ xelatex file.tex`
% \usepackage{fontspec}
% \setmonofont{M+ 1m}
% \setmonofont{M+ 1mn}
% \setmonofont{M+ 2m}

%-----------------------------------------------------------------------

% \usepackage{lmodern}
\usepackage{amssymb, amsmath}
\usepackage[makeroom]{cancel}
% \usepackage{ifxetex, ifluatex}
\usepackage{fixltx2e} % provides \textsubscript
\usepackage[utf8]{inputenc}
\usepackage[shorthands=off,main=brazil]{babel}
\usepackage{graphicx}
\usepackage{xcolor}
\usepackage{setspace}
\usepackage{comment}
\usepackage{icomma}

%-----------------------------------------------------------------------
% Algumas configurações.

\setlength{\parindent}{0pt}
\setlength{\parskip}{6pt plus 2pt minus 1pt}
\setlength{\emergencystretch}{3em}  % prevent overfull lines
% \providecommand{\tightlist}{%
%   \setlength{\itemsep}{0pt}\setlength{\parskip}{0pt}}
\setcounter{secnumdepth}{0}

% Espaço vertical para o ambiente `quote`.
\let\oldquote\quote
\let\oldendquote\endquote
\renewenvironment{quote}{%
  \vspace{1em}\oldquote}{%
  \oldendquote\vspace{1em}}

%-----------------------------------------------------------------------
% Espaçamento entre items para itemize, enumerate e description.

% % itemize.
% \let\itemopen\itemize
% \let\itemclose\enditemize
% \renewenvironment{itemize}{%
%   \itemopen\addtolength{\itemsep}{0.25\baselineskip}}{\itemclose}
%
% % enumerate.
% \let\enumopen\enumerate
% \let\enumclose\endenumerate
% \renewenvironment{enumerate}{%
%   \enumopen\addtolength{\itemsep}{0.25\baselineskip}}{\enumclose}
%
% % description.
% \let\descopen\description
% \let\descclose\enddescription
% \renewenvironment{description}{%
%   \descopen\addtolength{\itemsep}{0.25\baselineskip}}{\descclose}

%-----------------------------------------------------------------------

% \usepackage[hang]{caption}
\usepackage{caption}
\captionsetup{font=footnotesize,
  labelfont={color=mycolor1, footnotesize},
  labelsep=period}

% \providecommand{\tightlist}{%
%   \setlength{\itemsep}{0pt}\setlength{\parskip}{0pt}}

%-----------------------------------------------------------------------

\usepackage{tikz}

% \def\pathtobg{/home/walmes/Projects/templates/COMMON/ufpr-fundo.jpg}
% \def\pathtobg{/home/walmes/Projects/templates/COMMON/ufpr-fundo-16x9.jpg}
% \def\pathtobg{/home/walmes/Projects/templates/COMMON/ufpr-fachada-dir-1.jpg}
% \def\pathtobg{/home/walmes/Projects/templates/COMMON/ufpr-fachada-esq-1.jpg}
% \def\pathtobg{/home/walmes/Projects/templates/COMMON/ufpr-perto-1.jpg}
% \def\pathtobg{/home/walmes/Projects/templates/COMMON/ufpr-fachada-baixo-1.jpg}

\ifx\pathtobg\undefined
\else
  \usebackgroundtemplate{
    \tikz[overlay, remember picture]
    \node[% opacity=0.3,
          at=(current page.south east),
          anchor=south east,
          inner sep=0pt] {
            \includegraphics[height=\paperheight, width=\paperwidth]{\pathtobg}};
  }
\fi

%-----------------------------------------------------------------------
% Definições de esquema de cores.

\ifx\mycolorscheme\undefined
  % UFPR.
  % http://www.color-hex.com/color-palette/2018
  \definecolor{mycolor1}{HTML}{015c93} % Título.
  \definecolor{mycolor2}{HTML}{363435} % Texto.
  \definecolor{mycolor3}{HTML}{015c93} % Estrutura.
  \definecolor{mycolor4}{HTML}{015c93} % Links.
  \definecolor{mycolor5}{HTML}{CECAC5} % Preenchimentos.
\else
  \if\mycolorscheme1
    % UFPR.
    \definecolor{mycolor1}{HTML}{015c93} % Título.
    \definecolor{mycolor2}{HTML}{363435} % Texto.
    \definecolor{mycolor3}{HTML}{015c93} % Estrutura.
    \definecolor{mycolor4}{HTML}{015c93} % Links.
    \definecolor{mycolor5}{HTML}{CECAC5} % Preenchimentos.
  \fi
  \if\mycolorscheme2
    % DEST.
    \definecolor{mycolor1}{HTML}{2a0e72} % Título.
    \definecolor{mycolor2}{HTML}{202E35} % Texto.
    \definecolor{mycolor3}{HTML}{2a0e72} % Estrutura.
    % \definecolor{mycolor3}{HTML}{8072a3} % Estrutura.
    \definecolor{mycolor4}{HTML}{2a0e72} % Links.
    % \definecolor{mycolor4}{HTML}{bfb9d1} % Links.
    % \definecolor{mycolor5}{HTML}{AEA79F} % Preenchimentos.
    \definecolor{mycolor5}{HTML}{CECAC5} % Preenchimentos.
  \fi
  \if\mycolorscheme3
    % LEG.
    \definecolor{mycolor2}{HTML}{363435} % Texto.
    % \definecolor{mycolor1}{HTML}{ff8000} % Título.
    % \definecolor{mycolor3}{HTML}{ff8000} % Estrutura.
    % \definecolor{mycolor4}{HTML}{ff8000} % Links.
    % \definecolor{mycolor1}{HTML}{E57300} % Título.
    % \definecolor{mycolor3}{HTML}{E57300} % Estrutura.
    % \definecolor{mycolor4}{HTML}{E57300} % Links.
    \definecolor{mycolor1}{HTML}{F67014} % Título.
    \definecolor{mycolor3}{HTML}{F67014} % Estrutura.
    \definecolor{mycolor4}{HTML}{F67014} % Links.
    % \definecolor{mycolor1}{HTML}{FE5C23} % Título.
    % \definecolor{mycolor3}{HTML}{FE5C23} % Estrutura.
    % \definecolor{mycolor4}{HTML}{FE5C23} % Links.
    \definecolor{mycolor5}{HTML}{222222} % Preenchimentos.
    \definecolor{mycolor5}{HTML}{383838} % Preenchimentos.
  \fi
\fi

\hypersetup{
  colorlinks=true,
  linkcolor=mycolor4,
  urlcolor=mycolor1,
  citecolor=mycolor1
}

%-----------------------------------------------------------------------
% ATTENTION: http://www.cpt.univ-mrs.fr/~masson/latex/Beamer-appearance-cheat-sheet.pdf

\usetheme{Boadilla}
\usecolortheme{default}

% \setbeamersize{text margin left=7mm, text margin right=7mm}
% \setbeamertemplate{frametitle}[default][left, leftskip=3mm]
% \addtobeamertemplate{frametitle}{\vspace{0.5em}}{}

\setbeamertemplate{caption}[numbered]
\setbeamertemplate{section in toc}[sections numbered]
\setbeamertemplate{subsection in toc}[subsections numbered]
\setbeamertemplate{sections/subsections in toc}[ball]{}
\setbeamertemplate{sections in toc}[ball]
\setbeamercolor{section number projected}{bg=mycolor1, fg=white}
\setbeamertemplate{blocks}[rounded]
\setbeamertemplate{navigation symbols}{}
\setbeamertemplate{frametitle continuation}{\gdef\beamer@frametitle{}}
% \setbeamertemplate{frametitle}[default][center]
% \setbeamertemplate{footline}[frame number]

\setbeamertemplate{enumerate items}[default]
\setbeamertemplate{itemize items}{\scriptsize\raise1.25pt\hbox{\donotcoloroutermaths$\blacktriangleright$}}

% Blocos.
% \addtobeamertemplate{block begin}{\vskip -\bigskipamount}{}
% \addtobeamertemplate{block end}{}{\vskip -\bigskipamount}
\addtobeamertemplate{block begin}{\vspace{0.5em}}{}
\addtobeamertemplate{block end}{}{\vspace{0.5em}}


% Rodapé.
\setbeamercolor{title in head/foot}{parent=subsection in head/foot}
\setbeamercolor{author in head/foot}{bg=mycolor4, fg=white}
\setbeamercolor{date in head/foot}{parent=subsection in head/foot, fg=mycolor3}

% Cabeçalho.
\setbeamercolor{section in head/foot}{bg=mycolor2, fg=mycolor4}
\setbeamercolor{subsection in head/foot}{bg=mycolor2, fg=white}

\setbeamercolor{title}{fg=mycolor1}       % Título dos slides.
\setbeamercolor{titlelike}{fg=title}
\setbeamercolor{subtitle}{fg=mycolor2}    % Subtítulo.
\setbeamercolor{institute in head/foot}{parent=palette primary} % Instituição.
\setbeamercolor{frametitle}{fg=mycolor1}  % De quadro.
\setbeamercolor{structure}{fg=mycolor3}   % Listas e rodapé.
\setbeamercolor{item projected}{bg=mycolor2}
\setbeamercolor{block title}{bg=mycolor5, fg=mycolor2}
\setbeamercolor{normal text}{fg=mycolor2} % Texto.
\setbeamercolor{caption name}{fg=normal text.fg}
% \setbeamercolor{footlinecolor}{fg=mycolor2, bg=mycolor5}
% \setbeamercolor{section in head/foot}{fg=mycolor2, bg=mycolor5}
\setbeamercolor{author in head/foot}{fg=white, bg=mycolor1}
\setbeamercolor{section in foot}{fg=mycolor4, bg=mycolor5}
\setbeamercolor{date in foot}{fg=mycolor4, bg=mycolor5}
\setbeamercolor{block title}{fg=white, bg=mycolor1}
\setbeamercolor{block body}{fg=black, bg=white!80!gray}
\setbeamercolor{block body}{fg=black, bg=white!80!gray}

% To remove empty brackets of \institution.
\makeatletter
\setbeamertemplate{footline}{
  \leavevmode%
  \hbox{%
    \begin{beamercolorbox}[
      wd=0.3\paperwidth, ht=2.25ex, dp=1ex, right]{author in head/foot}%
      \usebeamerfont{author in head/foot}\insertshortauthor{}\hspace*{1ex}
    \end{beamercolorbox}%
    \begin{beamercolorbox}[
      wd=0.6\paperwidth, ht=2.25ex, dp=1ex, left]{section in foot}%
      \usebeamerfont{title in head/foot}\hspace*{1ex}\insertshorttitle{}
      % \usebeamerfont{title in head/foot}\hspace*{1ex}\insertframetitle{}
    \end{beamercolorbox}%
    \begin{beamercolorbox}[
      wd=0.1\paperwidth, ht=2.25ex, dp=1ex, right]{date in foot}%
      \insertframenumber{}\hspace*{2ex}
    \end{beamercolorbox}
  }%
  \vskip0pt%
}
\makeatother

%-----------------------------------------------------------------------

% \usepackage{hyphenat}
\usepackage{changepage}

% Slide para o título das seções.
\AtBeginSection[]{
  \begin{frame}
    % \vfill
    \vspace{4cm}
    % \centering
    % \begin{beamercolorbox}[sep = 8pt, center, shadow = true, rounded = true]{title}
    \begin{beamercolorbox}{title}
      \begin{columns}
        \column{0.7\linewidth}
        {\LARGE\textbf \insertsectionhead}
      \end{columns}
    \end{beamercolorbox}
    \vfill
  \end{frame}
}

%-----------------------------------------------------------------------
%---- preamble-chunk.tex -----------------------------------------------

% Knitr.

% ATTENTION: this needs `\usepackage{xcolor}'.
\definecolor{color_line}{HTML}{333333}
\definecolor{color_back}{HTML}{DDDDDD}
% \definecolor{color_back}{HTML}{FF0000}

% ATTENTION: usa o fancyvrb.
% https://ctan.math.illinois.edu/macros/latex/contrib/fancyvrb/doc/fancyvrb-doc.pdf
% R input.
\usepackage{tcolorbox}
\ifcsmacro{Highlighting}{
  % Statment if it exists. ------------------
  \DefineVerbatimEnvironment{Highlighting}{Verbatim}{
    % frame=lines,     % Linha superior e inferior.
    % framerule=0.5pt, % Espessura da linha.
    framesep=2ex,    % Distância da linha para o texto.
    % rulecolor=\color{color_line},
    % numbers=right,
    fontsize=\footnotesize, % Tamanho da fonte.
    baselinestretch=0.8,    % Espaçamento entre linhas.
    commandchars=\\\{\}}
  % Margens do ambiente `Shaded'.
  % \fvset{listparameters={\setlength{\topsep}{-1em}}}
  % \renewenvironment{Shaded}{\vspace{-1ex}}{\vspace{-2ex}}
  \renewenvironment{Shaded}{
    \vspace{2pt}
    \begin{tcolorbox}[
      boxrule=0pt,      % Espessura do contorno.
      colframe=gray!10, % Cor do contorno.
      colback=gray!10,  % Cor de fundo da caixa.
      arc=1em,          % Raio para contornos arredondados.
      sharp corners,
      boxsep=0.5em,     % Margem interna.
      left=3pt, right=3pt, top=3pt, bottom=3pt, % Margens internas.
      grow to left by=0mm,
      grow to right by=6pt,
      ]
    }{
    \end{tcolorbox}
    \vspace{-3pt}
    }
  }{
  % Statment if it not exists. --------------
}

% R output e todo `verbatim'.
\makeatletter
\def\verbatim@font{\linespread{0.8}\ttfamily\footnotesize}
%\makeatother

% Cor de fundo e margens do `verbatim'.
\let\oldv\verbatim
\let\oldendv\endverbatim

\def\verbatim{%
  \par\setbox0\vbox\bgroup % Abre grupo.
  %\vspace{-5px}            % Reduz margem superior.
  \oldv                    % Chama abertura do verbatim.
}
\def\endverbatim{%
  \oldendv                 % Chama encerramento do verbatim.
  %\vspace{0cm}           % Controla margem inferior.
  \egroup%\fboxsep5px      % Fecha grupo.
  \noindent{{\usebox0}}\par
}

%-----------------------------------------------------------------------
%---- preamble-commands.tex --------------------------------------------

% Para fazer texto em duas colunas.
\newcommand{\mytwocolumns}[4]{
  % #1: Line width fraction for the left column , e.g. 0.5.
  % #2: Line width fraction for the right column.
  % #3: Content for the left column.
  % #4: Content for the right column.
  \begin{columns}[c]
    \begin{column}{#1\linewidth} %----------- left.
      #3
    \end{column} %--------------------------- left.
    \begin{column}{#2\linewidth} %----------- right.
      #4
    \end{column} %--------------------------- right.
  \end{columns}
}

%-----------------------------------------------------------------------
% Para fazer duas colunas no Rmd.

% Center vertical align.
\def\beginAHalfColumn{\begin{minipage}{0.49\textwidth}}%
\def\beginAlmostHalfColumn{\begin{minipage}{0.45\textwidth}}%
\def\beginAQuarterColumn{\begin{minipage}{0.23\textwidth}}%
\def\beginThreeQuartersColumn{\begin{minipage}{0.72\textwidth}}%
\def\beginAThirdColumn{\begin{minipage}{0.31\textwidth}}%
\def\beginTwoThirdsColumn{\begin{minipage}{0.64\textwidth}}%
\def\endColumns{\end{minipage}}%

% Top vertical align.
\def\beginAHalfColumnT{\begin{minipage}[t]{0.49\textwidth}}%
\def\beginAlmostHalfColumnT{\begin{minipage}[t]{0.45\textwidth}}%
\def\beginAQuarterColumnT{\begin{minipage}[t]{0.23\textwidth}}%
\def\beginThreeQuartersColumnT{\begin{minipage}[t]{0.72\textwidth}}%
\def\beginAThirdColumnT{\begin{minipage}[t]{0.31\textwidth}}%
\def\beginTwoThirdsColumnT{\begin{minipage}[t]{0.64\textwidth}}%

%---------------------------------------------------------------------
% Ambientes para frases como e sem imagem.

\newcommand{\myquote}[3]{
  % #1: caminho para a imagem.
  % #2: a frase/quotation.
  % #3: o autor.
  \begin{center}
    \begin{minipage}[c]{0.19\linewidth}
      \begin{center}
        \includegraphics[height=2.5cm]{#1}
      \end{center}
    \end{minipage}
    \begin{minipage}[c]{0.7\linewidth}
      \begin{flushright}
        \textit{#2}
        \vspace{1ex}

        -- #3
      \end{flushright}
    \end{minipage}
  \end{center}
}

\newcommand{\myphrase}[2]{
  % #1: a frase/quotation.
  % #2: o autor.
  \begin{center}
    \begin{minipage}[c]{0.19\linewidth}
    \end{minipage}
    \begin{minipage}[c]{0.7\linewidth}
      \begin{flushright}
        \textit{#1}
        \vspace{1ex}

        -- #2
      \end{flushright}
    \end{minipage}
  \end{center}
}

%-----------------------------------------------------------------------
% Comandos para texto em destaque.

% \newcommand{\hi}[1]{%
%   \textcolor{ubuntu_orange}{#1}\xspace
% }

\usepackage{xspace}

% URLs com letra miuda.
\newcommand{\myurl}[1]{%
  {\tiny \url{#1}}\xspace
}

% Botões.
\newcommand{\btn}[1]{%
  \beamergotobutton{#1}\xspace
}

% Texto grande centralizado.
\newcommand{\centertitle}[1]{%
  \begin{center}
    {\LARGE \bfseries \hi{#1}}
  \end{center}
}

%-----------------------------------------------------------------------
\ifLuaTeX
  \usepackage{selnolig}  % disable illegal ligatures
\fi
\IfFileExists{bookmark.sty}{\usepackage{bookmark}}{\usepackage{hyperref}}
\IfFileExists{xurl.sty}{\usepackage{xurl}}{} % add URL line breaks if available
\urlstyle{same} % disable monospaced font for URLs
\hypersetup{
  pdfauthor={Prof.~Me. Lineu Alberto Cavazani de Freitas},
  hidelinks,
  pdfcreator={LaTeX via pandoc}}

\title{\textbf{Introdução}}
\subtitle{R, história, pacotes, RStudio e primeiro contato}
\author{Prof.~Me. Lineu Alberto Cavazani de Freitas}
\date{}
\institute{Métodos Estatísticos em Pesquisa Científica\\
Apoio computacional em linguagem R\\
\strut \\
Departamento de Estatística\\
Laboratório de Estatística e Geoinformação}

\begin{document}
\frame{\titlepage}

\begin{frame}{Estatística e o desenvolvimento computacional}
\protect\hypertarget{estatuxedstica-e-o-desenvolvimento-computacional}{}
\beginAHalfColumn

\begin{itemize}
\item
  A popularização da Estatística se dá graças ao
  \textbf{desenvolvimento computacional}.
\item
  Os computadores pessoais tornaram os métodos estatísticos mais
  acessíveis ao público geral por meio de \textbf{softwares} que
  implementam as metodologias.
\item
  Devido ao avanço computacional, houve um aumento considerável na
  capacidade de produzir e armazenar dados provenientes das mais
  diversas fontes.
\end{itemize}

\endColumns
\beginAHalfColumn

\begin{figure}

{\centering \includegraphics[width=0.6\linewidth]{./img/desenvolvimento-computacional} 

}

\caption{Extraído de \href{https://cdn.pixabay.com/photo/2020/04/04/04/23/graph-5000784_1280.png}{pixabay.com.}}\label{fig:unnamed-chunk-2}
\end{figure}

\endColumns
\end{frame}

\begin{frame}{Estatística e o desenvolvimento computacional}
\protect\hypertarget{estatuxedstica-e-o-desenvolvimento-computacional-1}{}
\beginAHalfColumn

\begin{itemize}
\item
  Graças ao avanço computacional podemos lidar com a manipulação de
  \textbf{grandes conjuntos de dados}.
\item
  Este grande volume de dados também força o
  \textbf{desenvolvimento dos métodos estatísticos} e softwares para
  análise de dados.
\item
  A capacidade computacional atual também desperta o interesse por
  \textbf{métodos estatísticos computacionalmente intensivos}.
\end{itemize}

\endColumns
\beginAHalfColumn

\begin{figure}

{\centering \includegraphics[width=0.6\linewidth]{./img/ti} 

}

\caption{Extraído de \href{https://cdn.pixabay.com/photo/2015/04/14/23/17/it-business-722950_1280.png}{pixabay.com.}}\label{fig:unnamed-chunk-3}
\end{figure}

\endColumns
\end{frame}

\begin{frame}{Ferramentas para análises estatísticas}
\protect\hypertarget{ferramentas-para-anuxe1lises-estatuxedsticas}{}
Existem diversas ferramentas disponíveis:

\beginAHalfColumn

\begin{itemize}
\item
  R;
\item
  Python;
\item
  SAS;
\item
  Spss;
\item
  Biostat;
\item
  Minitab;
\item
  Tableau;
\item
  Stata;
\item
  E diversas outras.
\end{itemize}

\endColumns
\beginAHalfColumn

\begin{figure}

{\centering \includegraphics[width=0.8\linewidth]{./img/programacao} 

}

\caption{Extraído de \href{https://cdn.pixabay.com/photo/2018/06/08/00/48/developer-3461405_960_720.png}{pixabay.com.}}\label{fig:unnamed-chunk-4}
\end{figure}

\endColumns
\end{frame}

\begin{frame}{Ferramentas para análises estatísticas}
\protect\hypertarget{ferramentas-para-anuxe1lises-estatuxedsticas-1}{}
Existem diversas ferramentas disponíveis:

\beginAHalfColumn

\begin{itemize}
\item
  \textbf{R};
\item
  Python;
\item
  SAS;
\item
  Spss;
\item
  Biostat;
\item
  Minitab;
\item
  Tableau;
\item
  Stata;
\item
  E diversas outras.
\end{itemize}

\endColumns
\beginAHalfColumn

\begin{figure}

{\centering \includegraphics[width=0.8\linewidth]{./img/rlogo} 

}

\caption{Logo do R.}\label{fig:unnamed-chunk-5}
\end{figure}

\endColumns
\end{frame}

\begin{frame}{R}
\protect\hypertarget{r}{}
\begin{itemize}
\item
  R é uma linguagem e ambiente para \textbf{computação estatística} e
  \textbf{gráficos}.
\item
  É \textbf{livre} e de \textbf{código aberto}.

  \begin{itemize}
  \tightlist
  \item
    Livre (free): usuários tem liberdade de:

    \begin{enumerate}
    \tightlist
    \item
      executar como desejar e para qualquer propósito.
    \item
      estudar o funcionamento e adapta-lo à necessidades específicas.
    \item
      distribuir cópias de versões originais e modificadas.
    \end{enumerate}
  \item
    Código aberto (open source): o acesso ao código fonte é gratuito.
  \end{itemize}
\end{itemize}
\end{frame}

\begin{frame}{R}
\protect\hypertarget{r-1}{}
\begin{itemize}
\tightlist
\item
  Muito popular no meio \textbf{acadêmico} e tem uso cada vez maior no
  meio \textbf{corporativo}.

  \begin{itemize}
  \tightlist
  \item
    É acessível e gratuito.
  \item
    Tem diversas técnicas e aplicações possíveis.
  \end{itemize}
\item
  Tem potencial uso em todas as etapas do processo de análise de dados.

  \begin{itemize}
  \tightlist
  \item
    Obtenção, importação, manipulação e tratamento.
  \item
    Análise exploratória.
  \item
    Ajuste de modelos estatísticos, modelos de aprendizado de máquina,
    dentre outros.
  \item
    Elaboração de relatórios dinâmicos e reproduzíveis.
  \end{itemize}
\end{itemize}
\end{frame}

\begin{frame}{Antes do R}
\protect\hypertarget{antes-do-r}{}
\begin{itemize}
\item
  O R sucedeu a lingugem S, de \textbf{John Chambers}, Rick Becker,
  Trevor Hastie, Allan Wilks e outros.
\item
  A linguagem S foi iniciada em 1976 e disponibilizada publicamente no
  início dos anos 80 já como um ambiente de análise estatística.
\item
  Uma das principais limitações da linguagem S era que ela só estava
  disponível em um \textbf{pacote comercial}: o S-PLUS. O que motivou a
  criação do R.
\item
  Quando o R foi desenvolvido a sintaxe era muito semelhante ao S,
  pensando na migração dos usuários de uma linguagem para outra.
\end{itemize}
\end{frame}

\begin{frame}{História do R}
\protect\hypertarget{histuxf3ria-do-r}{}
\beginAHalfColumn

\begin{itemize}
\item
  Em \textbf{1991} o R foi criado por Ross Ihaka e Robert Gentleman
  (``\textbf{R \& R}'') no Departamento de Estatística da Universidade
  de Auckland.
\item
  Em \textbf{1993} o R foi \textbf{anunciado publicamente} pela primeira
  vez.
\item
  Em \textbf{1995} o R passou a usar a
  \textbf{GNU General Public License}, o que tornou o R um software
  livre e de código aberto.
\end{itemize}

\endColumns
\beginAHalfColumn

\begin{figure}

{\centering \includegraphics[width=0.8\linewidth]{./img/r&r} 

}

\caption{Robert Gentleman e Ross Ihaka.}\label{fig:unnamed-chunk-6}
\end{figure}

\endColumns
\end{frame}

\begin{frame}{História do R}
\protect\hypertarget{histuxf3ria-do-r-1}{}
\beginAHalfColumn

\begin{itemize}
\item
  Em \textbf{1996} foi publicado o \textbf{artigo} em que Ross e Robert
  descrevem sua proposta:
  ``\emph{R: A language for data analysis and graphics.}'' no Journal of
  Computational and Graphical Statistics.
\item
  Ainda em \textbf{1996} foram criadas as listas de discussão
  \textbf{R-help} e \textbf{R-devel}.
\end{itemize}

\endColumns
\beginAHalfColumn

\begin{figure}

{\centering \includegraphics[width=0.8\linewidth]{./img/r-paper} 

}

\caption{Artigo original do R.}\label{fig:unnamed-chunk-7}
\end{figure}

\endColumns
\end{frame}

\begin{frame}{História do R}
\protect\hypertarget{histuxf3ria-do-r-2}{}
\begin{itemize}
\item
  Em \textbf{1997} foi formado o \textbf{R Core Team}: um grupo de
  aproximadamente 20 desenvolvedores que mantêm, gerenciam, controlam o
  código fonte e orientam a evolução da linguagem.
\item
  Os membros do R Core Team fundaram a \textbf{R Foundation}: uma
  organização sem fins lucrativos que trabalha no interesse público para
  dar suporte ao R.
\item
  Os direitos autorais do código-fonte primário do R pertencem à R
  Foundation e são publicados sob a GNU General Public License versão
  2.0.
\end{itemize}
\end{frame}

\begin{frame}{História do R}
\protect\hypertarget{histuxf3ria-do-r-3}{}
\beginAHalfColumn

\begin{itemize}
\item
  Em \textbf{2000} o R 1.0.0 foi lançado.
\item
  O R base está disponível para instalação no
  \textbf{Comprehensive R Archive Network}, também conhecido como
  \textbf{CRAN}.
\end{itemize}

\endColumns
\beginAHalfColumn

\begin{figure}

{\centering \includegraphics[width=0.8\linewidth]{./img/r110} 

}

\caption{R 1.0.0.}\label{fig:unnamed-chunk-8}
\end{figure}

\endColumns
\end{frame}

\begin{frame}{O que é o R}
\protect\hypertarget{o-que-uxe9-o-r}{}
\begin{itemize}
\item
  O R é uma \textbf{linguagem de programação}.
\item
  Você faz a análise de dados escrevendo \textbf{funções e scripts}, não
  apontando, clicando e arrastando caixas.
\item
  Para quem nunca programou, parece assustador. Mas o R é fácil de
  aprender e guiado a análise de dados.
\item
  É possível instalar e usar o R nos principais sistemas operacionais.
\item
  Assim como vários outros softwares livres e de código aberto, o R tem
  lançamentos frequentes de \textbf{versões}.
\item
  A comunidade R é altamente ativa com usuários no mundo todo que
  contribuem, desenvolvem pacotes e ajudam uns aos outros por meio de
  materiais online como listas de discussão e tutoriais.
\end{itemize}
\end{frame}

\begin{frame}{R e os pacotes}
\protect\hypertarget{r-e-os-pacotes}{}
\beginAHalfColumn

\begin{itemize}
\item
  \textbf{Pacotes R} são coleções de funções R, dados e código
  compilado.
\item
  O R já vem com um conjunto de pacotes por padrão e outros podem ser
  adicionados para estender os recursos.
\item
  Os pacotes hoje disponíveis são o resultado de anos de colaboração de
  pessoas de todo o mundo.
\item
  Uma das funções do CRAN é hospedar diversos pacotes complementares.
\end{itemize}

\endColumns
\beginAHalfColumn

\begin{figure}

{\centering \includegraphics[width=0.6\linewidth]{./img/pacote} 

}

\caption{Extraído de \href{https://cdn.pixabay.com/photo/2014/12/21/23/35/parcel-575623_960_720.png}{pixabay.com.}}\label{fig:unnamed-chunk-9}
\end{figure}

\endColumns
\end{frame}

\begin{frame}[fragile]{R-base}
\protect\hypertarget{r-base}{}
\begin{itemize}
\item
  O R ``base'' contém o pacote básico necessário para executar o R e as
  funções mais fundamentais.
\item
  As funções já vem disponíveis, prontas para chamada e uso.
\item
  É composto por \textbf{15} pacotes:
\end{itemize}

\beginAHalfColumn

\begin{enumerate}
\tightlist
\item
  \texttt{base}
\item
  \texttt{compiler}
\item
  \texttt{datasets}
\item
  \texttt{grDevices}
\item
  \texttt{graphics}
\item
  \texttt{grid}
\item
  \texttt{methods}
\item
  \texttt{parallel}
\end{enumerate}

\endColumns
\beginAHalfColumn

\begin{enumerate}
\setcounter{enumi}{8}
\tightlist
\item
  \texttt{splines}
\item
  \texttt{stats}
\item
  \texttt{stats4}
\item
  \texttt{tcltk}
\item
  \texttt{tools}
\item
  \texttt{translations}
\item
  \texttt{utils}.
\end{enumerate}

\endColumns
\end{frame}

\begin{frame}[fragile]{R-recommended}
\protect\hypertarget{r-recommended}{}
\begin{itemize}
\item
  Um segundo grupo de pacotes que já vem com a instalação do R são os
  pacotes ``recomendados''.
\item
  Apesar de já instaladas, estas bibliotecas precisam ser chamadas para
  que seja possível usar as funções.
\item
  É composto por outros \textbf{15} pacotes.
\end{itemize}

\beginAHalfColumn

\begin{enumerate}
\tightlist
\item
  \texttt{KernSmooth}
\item
  \texttt{MASS}
\item
  \texttt{Matrix}
\item
  \texttt{boot}
\item
  \texttt{class}
\item
  \texttt{cluster}
\item
  \texttt{codetools}
\item
  \texttt{foreign}
\end{enumerate}

\endColumns
\beginAHalfColumn

\begin{enumerate}
\setcounter{enumi}{8}
\tightlist
\item
  \texttt{lattice}
\item
  \texttt{mgcv}
\item
  \texttt{nlme}
\item
  \texttt{nnet}
\item
  \texttt{rpart}
\item
  \texttt{spatial}
\item
  \texttt{survival}.
\end{enumerate}

\endColumns
\end{frame}

\begin{frame}{Outros pacotes}
\protect\hypertarget{outros-pacotes}{}
\beginAHalfColumn

\begin{itemize}
\item
  Você pode facilmente obter e instalar pacotes além dos 30 que já vem
  com a instalação tradicional do R.
\item
  A principal fonte de pacotes é o próprio CRAN, que hoje conta com
  \textbf{mais de 19000 pacotes}.
\item
  Fontes secundárias envolvem páginas web e repositórios como github,
  onde desenvolvedores mantém pacotes em desenvolvimento.
\end{itemize}

\endColumns
\beginAHalfColumn

\begin{figure}

{\centering \includegraphics[width=0.9\linewidth]{./img/list-packages} 

}

\caption{Lista de pacotes disponíveis por nome no CRAN.}\label{fig:unnamed-chunk-10}
\end{figure}

\endColumns
\end{frame}

\begin{frame}{Limitações}
\protect\hypertarget{limitauxe7uxf5es}{}
\begin{itemize}
\item
  O R é essencialmente baseado em tecnologia antiga (sistema S da Bell
  Labs).
\item
  Por ser baseado em tecnologia antiga o R sofre com suporte para
  algumas ferramentas modernas como gráficos dinâmicos.
\item
  O R demanda memória física, os objetos geralmente são armazenados em
  memória. É algo que vem sendo melhorado aos poucos.
\item
  O R depende dos seus usuários. Os novos recursos dependem do interesse
  da comunidade. A partir do momento em que o interesse acabar, o R
  ficará estagnado.
\end{itemize}
\end{frame}

\begin{frame}{Em resumo}
\protect\hypertarget{em-resumo}{}
O R fornece

\begin{itemize}
\tightlist
\item
  Diversos recursos de Estatística.
\item
  Diversos recursos gráficos.
\item
  Uma vasta coleção de pacotes oficiais e não oficiais.
\item
  Uma linguagem de programação bem desenvolvida, simples e eficaz.
\item
  Possibilidade de instalação e uso em uma ampla variedade de
  plataformas UNIX e sistemas similares (incluindo FreeBSD e Linux),
  Windows e MacOS.
\item
  Possibilidade de uso de códigos C, C++ e Fortran para tarefas
  computacionalmente intesivas.
\item
  Documentação padronizada.
\end{itemize}
\end{frame}

\begin{frame}{IDEs e Editores}
\protect\hypertarget{ides-e-editores}{}
\beginAHalfColumn

\begin{itemize}
\item
  Existem softwares adicionais úteis para ajudar a programar de forma
  mais rápida e eficiente.
\item
  As IDE's (\emph{Integrated Development Environment}) são softwares que
  oferecem algumas facilidades para se programar em determinada
  linguagem.
\item
  Já os editores tendem a ser úteis para múltiplas linguagens e fornecem
  mais alternativas de customização.
\end{itemize}

\endColumns
\beginAHalfColumn

\begin{itemize}
\tightlist
\item
  Para trabalhar em R, dentre IDEs e editores, o RStudio IDE é a opção
  mais famosa.
\end{itemize}

\vspace{1cm}

\begin{figure}

{\centering \includegraphics[width=0.6\linewidth]{./img/rstudiologo} 

}

\caption{Logo do RStudio.}\label{fig:unnamed-chunk-11}
\end{figure}

\endColumns
\end{frame}

\begin{frame}{Links}
\protect\hypertarget{links}{}
\beginAHalfColumn

\begin{itemize}
\item
  \href{https://cran.r-project.org/}{\color{blue}{The Comprehensive R Archive Network}}.
\item
  \href{https://posit.co/download/rstudio-desktop/}{\color{blue}{RStudio Desktop}}.
\item
  \href{https://posit.cloud/plans/free}{\color{blue}{Posit Cloud}}.
\end{itemize}

\endColumns
\beginAHalfColumn

\endColumns
\end{frame}

\begin{frame}{Materiais para aprender R}
\protect\hypertarget{materiais-para-aprender-r}{}
\begin{itemize}
\item
  Mayer, FP; Bonat, WH; Zeviani, WM; Krainski, EK; Ribeiro Jr., PJ.
  \href{http://cursos.leg.ufpr.br/ecr/}{\color{blue}{Estatística Computacional com R}}.
  DEST/UFPR, 2018.
\item
  Ribeiro Jr., PJ.
  \href{http://www.leg.ufpr.br/~paulojus/embrapa/Rembrapa/}{\color{blue}{Introdução ao Ambiente Estatístico R}}.
  2011.
\item
  Horton, NJ; Pruim, R; Kaplan, DT.
  \href{http://cran-r.c3sl.ufpr.br/doc/contrib/Horton+Pruim+Kaplan_MOSAIC-StudentGuide.pdf}{\color{blue}{A Student's Guide to R}}.
  2015.
\item
  Maindonald, JH.
  \href{http://cran-r.c3sl.ufpr.br/doc/contrib/usingR.pdf}{\color{blue}{Using R for Data Analysis and Graphics}}.
  2008.
\item
  Paradis, E.
  \href{http://cran-r.c3sl.ufpr.br/doc/contrib/Paradis-rdebuts_en.pdf}{\color{blue}{R for Beginners}}.
  2005.
\end{itemize}
\end{frame}

\begin{frame}{Referências}
\protect\hypertarget{referuxeancias}{}
PENG, Roger D.
\href{https://bookdown.org/rdpeng/rprogdatascience/}{\color{blue}{R programming for data science}}.
Victoria, BC, Canada: Leanpub, 2016.

IHAKA, Ross.
\href{https://www.stat.auckland.ac.nz/~ihaka/downloads/Interface98.pdf}{\color{blue}{R: Past and future history}}.
Computing Science and Statistics, v. 392396, 1998.

Microsoft R Application Network.
\href{https://mran.microsoft.com/documents/what-is-r}{\color{blue}{What is R?}}

The R Project for Statistical Computing.
\href{https://www.r-project.org/about.html}{\color{blue}{What is R?}}
\end{frame}

\end{document}
